Code readability is closely related to defect detection problem. There are plenty of 
studies where the problem of code readability is considered 
\citep{8651396, xxx66666444, 10.1145/1985441.1985454}. At the moment, it is 
still the issue since the problem is too subjective. 

It is interesting that \citet{10.1109/ICPC.2019.00014} tried to evaluate 
the models for code readbility problem. The authors extracted 548 commits 
from 63 engineered Java software projects. The authors identified a commit as a readability commit
when the message of the commit directly indicated it. E.g., 
the authors searched such words of the commit as ``readable'',
``readability'', ``easier to read'', ``understand'', etc.
Several state of the art readability models were reviewed by \citet{10.1109/ICPC.2019.00014} 
and it was discovered, that the models failed to capture readability improvements. 
The authors also suggested several metrics which were not considered in the reviewed state-of-art papers.
The authors believe that those metrics for code readability can be used to detect readabilty 
changes more efficiently.

It is important to notice that code readabilty also can have a correlation 
with complexity of the code. \citet{10.1109/TSE.2009.70} compared code readability 
and software complexity. They used tools, published by \citet{xxx66666444, Readabil74:online}, to
compute metrics for code readability like \emph{The Automated Readability Index}, 
\emph{The Simple Measure of Gobbledygook}, \emph{Flesch-Kincaid Readability Index}
\emph{The Gunning's Fog Index}, \emph{Coleman-Liau Index and Buse Readability Score}. 
Also, they computed complexity software metrics, such as \emph{Halstead Complexity Volume},
\emph{McCabe's Cyclomatic Complexity}. The authors found out that there was a 
negative correlation between the readability and complexity metrics.
It means that low readability increases program complexity and vice versa. 
The authors also mentioned that the languages constructions as comments, spacing, 
while loop, meaningful names and do-while loop affects the code readiblity the most.
The authors also published a dataset with Java code which can help to detect defects.

\citet{10.1007/978-3-319-95171-3_32} proposed a model which quantitatively measures the readability 
of source code. The first idea is the using metrics (\emph{LOC, ProgramVolume, Entropy})
as the key indicators which affect the source code readability. The second
idea is that the authors introduced the equation for quantitative measure 
the source code readability in real time. This measure can demonstrate 
how small changes in the code affect the readability of the code.
The authors also tried to optimize the model and they reached about 
74.59\% of explanatory power. 
