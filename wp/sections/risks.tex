There is a number of risks we identified and expect to mitigate.

\textbf{Survey Validity}.
While designing the survey procedure we have to 
consider the following threats to its validity:
\begin{enumerate*}[label=\arabic*)]
    \item Set of interviewers must be representive. 
    Will the results be different if we take a different set of interviewers?
    \item Set of code snippets must be representive. 
    Will the results be different if we take a different set of code snippets? 
    \item How to identify bad interviewers? 
    Some interviewers may give answers that are not correlated with actual readability.
\end{enumerate*}
To mitigate this risk we have to think about the diversity of interviewers, 
for example by age, skills, experience, education, and so on. We have to select code snippets
that vary in length and project domain. The more data we collect the better. 
We have to preprocess survey results to exclude outliers.

\textbf{Feature Limits}.
To predict the readability we are going to use features 
calculated with static code analysis tools. 
These features mainly reflect the structure, the syntax and 
size properties of a particular code snippet.  
However, we are not considering semantic properties of the code. 
In the future, we can design semantic related features and 
add them to our dataset.