% The refactoring recommendation
% is a pointer to specific line code with description what to do to improve code readability. 
% To do this we studied a relationships between the code readability and mannualy designed AST patterns usually 
% encountered in the code. Knowing that the pattern presented in code and has negative impact on readability
% we do our recommendations, pointing to code line where this pattern presented in the code.

% In our work we created a dataset of Java code snipets features. The features may be devided into 
% three groups: code metrics, AST patterns and code readability. The code metrics calculated using existing
% tools: CheckStyle, RefactoringMiner, ChangeDistiller, and SourceMeter. The AST patterns are manually designed
% features that reflect a presense of some syntax structures or other static code preperties, for example
% number of nested FOR loops of length 2 or number ternary operators. To obtain the code readability we conducted 
% a survey that described in the next section.

% \subsection{Dataset}

% Next, we enriched each pair with metrics calculated 
% based on static code, for example \acrfull{sloc}, \acrfull{cyclo}. Eventually we 
% have the dataset, where for each snippet of code there are readability score and the set of metrics.


% % \begin{itemize}
% %     \item Is it possible to predict code readability using static metrics of code? 
% %     \item Which metrics do have more importance to predict code readability?
% %     \item Which ML model predict code readability better?
% %     \item Is it possible to recommend particular code refactoring to improve code readability?
% % \end{itemize}

% \subsection{Methodology}

% We split our work into two parts. The first one is readability score prediction. 
% The second part is code refactoring recommendations to improve code readability.

% \subsubsection{Readability prediction}

% Having unseen code snippet we have to estimate its readability score, because we are
% not going to give recomendation if the readability score is acceptable.
% We train a \acrshort{ml} model using gathered dataset.
% We stated the problem as a regression problem of predicting the readability score:

% $$
%     r_{i} = f(X_{i}, \theta) + e_{i}
% $$

% where $i$ is the index of rows in our dataset, $r$ is the readability score, $X$ is 
% the set of the code snippet's metrics, $e$ is some error or noise. And our goal is to choose 
% the parametric function $f$ and finds its parameters $\theta$.

% We tested a set of well-known regression models, like Linear Regression, \acrfull{cart}, 
% to predict readability. We selected the best model among 
% the considered set.

% \subsubsection{Refactoring recommendations}

% We measured a features importance with respect to readability score and made an ordered
% list of all features by importance.
% Having unseen code snippet with unacceptable readability score we do our refactoring recomendations.
% We rank our recommendations based on feature importance list. For AST related features we are able 
% to point to specific place in the code. 


% We formulate our business goal following way: having code snippet we want 
% get list of refactoring recommendations indicating line and what to do to improve code quality. 
% List should be ordered by refactoring impact on quality.