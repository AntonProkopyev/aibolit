% SPDX-FileCopyrightText: Copyright (c) 2019-2025 Aibolit
% SPDX-License-Identifier: MIT

Insufficient software quality may result in increased development costs and
negatively affect customer satisfaction \citep{The_Economics_of_Software_Quality}.
\textit{Static code analysis} develops techniques to help detect software quality
issues prior to program execution. It has practical applications in various developer tools. There are both open-source\footnote{PMD: \url{http://pmd.sourceforge.net/}, Rubocop: \url{https://github.com/rubocop-hq/rubocop},
PHPCS: \url{https://github.com/squizlabs/PHP_CodeSniffer},
FindSecBugs: \url{https://find-sec-bugs.github.io/}, ESLint: \url{https://eslint.org/}, Checkstyle: \url{https://checkstyle.sourceforge.io/}.}
(PMD, Rubocop, PHPCS, FindSecBugs, ESLint, Checkstyle, to name a few) and commercial\footnote{IBM Security AppScan: \url{https://www.hcltechsw.com/wps/portal/products/appscan},
PVS-Studio: \url{https://www.viva64.com/en/pvs-studio/},
SonarQube: \url{https://www.sonarqube.org/},
Parasoft: \url{https://www.parasoft.com/}}
(IBM Security AppScan, PVS-Studio, SonarQube, Parasoft) static analyzers
on the market.

Static code analysis can be applied to improve an \textit{internal} and an
\textit{external} quality of software \citep{Ilyas2016StaticCA}. External
quality is related to defects encountered by the end user of the software
product. Within internal quality, two important subcategories are
\textit{functional quality} and \textit{maintainability}. Functional quality is
about code correctness and compliance with the functional software
specifications \citep{Farhan}. Code maintainability is about how easy it is to
analyze, modify, and adapt given software \citep{Mohammadi2013AnAO}.

Functional quality aspects are typically quite susceptible to formal definition
and quantification.
% (\todo: examples!!).
Functional quality is also an essential
requirement in any domain of software development. On the other hand,
maintainability is a lot less straightforward to formally specify or quantify.
%\todo: refs.
Also, in certain applications it appears less important than
functional correctness, although in the business domain it is recognized as an
essential property.
% (\todo: ref).
As a result, there is currently a lot more
research and practical tools addressing functional quality aspects of code than
maintainability \citep{Overview_Static_Code_Analysis_in_Software_Development}.
Another aspect of static analysis tools that may have hindered their application
to maintainability, is that they are predominantly rule-based. Since there has
not yet been a consensus on how to formalize maintainability, it is challenging
to devise a set of formal rules to detect it.

We designed our new tool Aibolit to help developers identify patterns in their
code that may cause maintainability issues. It is a next-generation static
analysis tool that uses a machine learning (ML) model as an underlying quality
prediction mechanism. From the perspective of ML, our product is a recommender
system. For a given class file, it gives suggestions to the developer to alter
their code. The recommendations come in the form of \textit{code patterns} that
are detected in the code and advised to be removed.

Our choice to design Aibolit as an ML-based system alleviates some important
shortcomings of rule-based static analyzers. By design, ML algorithms capture
statistical relations in the external world (data). Therefore, they can be a
good way to model imprecisely and subjectively defined properties of code, such
as its maintainability. Moreover, rule-based systems are known to not scale well
to the diversity of empirically observed cases, and they tend to get very hard
to extend and maintain \citep{LenatFeigenbaum1987}. The ML
approach does not require
manual system adaptation as new observations or new features (patterns) come
along. In fact, Aibolit provides an easy way for developers to integrate a code
pattern of their liking into the recommender system and to analyze the pattern's
impact on code quality.






% This document explains how Aibolit works  and what makes it novel and
% different from other static analyzers (Section~\ref{sec:method}).
% We further show how Aibolit can be extended
% with custom patterns and metrics. We discuss the current shortcomings and the way
% they can be addressed in future work (Section~\ref{sec:conclusion}).
